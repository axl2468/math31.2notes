\documentclass{article}
\usepackage{amsmath}
\usepackage{amsfonts}
\usepackage{amssymb}
\usepackage{multicol}
\usepackage{graphicx}

\graphicspath{ {./images/} }

\setlength{\parindent}{0pt}

\begin{document}

\section*{03/18/2024}
\subsection*{M10: Mathematics in the Modern World}
\begin{enumerate}
    \item Terminologies
        \begin{enumerate}
            \item Data
             - collection of measured information. Can be qualitative or quantitative.
                \begin{enumerate}
                    \item Univariate - single measurement obtained
                    \item multivariate - multiple measurements obtained
                \end{enumerate}
            \item Variable - characteristic of an object or person
                \begin{enumerate}
                    \item Discrete - represented by integers
                    \item Continuous - represented in decimal form
                \end{enumerate}
            \item Data Management - practice of managing data as a valuable resource.
            \item Population vs Sample Descriptive Measures
                \begin{enumerate}
                    \item Population - described by parameters
                    \item Sample - described by statistics
                \end{enumerate}
        \end{enumerate}
    \item Sample Problems
        \begin{enumerate}
            \item Find the mean and median monthly income of 100 families if 95 of them earn 20,000 pesos per month and the rest earn 300,000 pesos per month. What can you infer from the values of the mean and median in this example?                    
                    \begin{align*}
                        sample mean&=\bar{x} \\
                        \bar{x}&=9T(20,000)+5(300,000)/(100) \\
                        \bar{x}&=34,200 && (\text{mean monthly income})
                    \end{align*}
                    \begin{align*}
                        sample median&=\tilde{x} \\
                        \tilde{x}&=20,000 && (\text{median monthly income})
                    \end{align*}
            \item Find the mean and median age on the table.
                    \begin{align*}
                        mean&=\bar{x} \\
                        \bar{x}&=\frac{7(3)+8(4)+9(6)+10(15)+11(11)+12(7)+13(1)}{3+4+6+15+11+7+1} \\
                        \bar{x}&=10
                    \end{align*}
            \item A teacher grades on 5 tests, a project, and a final exam. Each test counts as 10 percent of the course grade and the final exam counts as 30 percent of the course grade. Sam's test scores (in percent) are 70, 65, 82, 94, and 85. Her project score is 92 percent and her final score is 80 percent. Find Sam's average for the course.
                    \begin{align*}
                        weighted mean&=\bar{x} \\
                        \bar{x}&=(70+65+82+94+85)(0.10)+(92)(0.20)+(85)(0.30) \\
                        \bar{x}&=82
                    \end{align*}                  
            \item Two softdrink dispensing machines were designed to dispence 8 ounces of liquid into a cup. The amount of soft drink dispensed by the machine in 5 samples is recorded in the table below. How many ounces of softdrink are being dispensed by each machine? Give a valuable insight based on the data (on the table) and your computation.
                    \begin{align*}
                        Machine 1&=\bar{x_1} \\
                        \bar{x_1}&=\frac{(9.52)(6.41)(10.07)(5.85)(8.15)}{5} \\
                        \bar{x_1}&=8 
                    \end{align*}               
                    \begin{align*}
                        Machine 2&=\bar{x_2} \\
                        \bar{x_2}&=\frac{(8.01)(7.99)(7.95)(8.03)(8.02)}{5} \\
                        \bar{x_2}&=8 
                    \end{align*}                 
            \item Using the previous problem, find the range of the two machines.
                \begin{equation}
                    \begin{aligned}
                        Machine 1&=R_1 \\
                        R_1&=10.07-5.85 \\
                        R_1&=4.22
                    \end{aligned}               
                \end{equation}
                \begin{equation}
                    \begin{aligned}
                        Machine 2&=R_2 \\
                        R_2&=8.03-7.95 \\
                        R_2&=0.08
                    \end{aligned}               
                \end{equation}
        \end{enumerate}
    \item On Culminating Activity
        \begin{enumerate}
            \item Think of your groupmates (:heart emoji:)(:heart emoji:)(:heart emoji:)
            \item Think of a theme for your project similar to what sir will present.
            \item Think of a topic you can talk about.
            \begin{enumerate}
                \item Discuss about use of math in your topic.
                \item Discuss a problem related to your topic.
                \item Make a proposal to address the problem using data management tools.
            \end{enumerate}
        \end{enumerate}
\end{enumerate}
\subsection*{M31.2: Mathematical Analysis IB}
\begin{enumerate}
    \item Discussion
    \begin{enumerate}
        \item $\int f(sinx,cosx) \,dx = \int u \,du = F(u), u = sinx$\\or\\
        $\int f(sinx,cosx) \,dx = -\int u \,du = -F(u), u = cosx$ 
        \\\\Example:\\
        $\int {cos^5(x)sin^3(x)} \,dx $\\
        $\int {cos^5(x)(1-cos^2(x))sin(x)} \,dx = $\\
        $\int {u^5(1-u^2)} \,du = $\\
        $\int {u^5-u^7} \,du = {\frac{u^6}{6}-\frac{u^8}{8}}$\\\\
        \item $\int {sec^3(x)} \,dx $\\
        Let: $u=secx;du=sectanxdx;dv=sec^2(x)dx;v=tanx$\\
        $\int {sec^3(x)} \,dx = secxtanx - \int {sec(x)tan^2(x)} \,dx $\\
        $\int {sec^3(x)} \,dx = secxtanx - \int {sec(x)(sec^2(x)-1)} \,dx $\\
        $\int {sec^3(x)} \,dx = secxtanx - \int {sec^3(x)-sec(x)} \,dx $\\
        $2\int {sec^3(x)} \,dx = secxtanx + \int {sec(x)} \,dx $\\
        $2\int {sec^3(x)} \,dx = secxtanx + ln|secx+tanx| \,dx $\\
        $\int {sec^3(x)} \,dx = \frac{secxtanx + ln|secx+tanx|}{2} \,dx $\\\\
        \item $\int f(x\sqrt[2]{a^2-x^2}) \,dx; x=asin\theta or x=acos\theta$\\\\
        Example: $\int {\frac{x^2}{\sqrt{1-x^2}}} \,dx$\\
        $\int {\frac{sin^2(\theta)}{cos\theta}cos\theta} \,d\theta$\\

        \item $\int f(x\sqrt[2]{a^2+x^2}) \,dx; x=atan\theta$\\
        \item $\int f(x\sqrt[2]{x^2-a^2}) \,dx; x=asec\theta$\\
        \item $\int(sin^2(\theta)) \,d\theta$\\\\\\
        \item cos
        
        \item $\int sin^2(x)cos^2(x) \,dx$\\
        $\int {(\frac{1+cos2x}{2})(\frac{1-cos2x}{2})} \,dx$\\
        $\int {(\frac{1-cos^2(2x)}{4})} \,dx$\\
        $\int {\frac{1}{4}} \,dx - \int {\frac{cos^2(2x)}{4}} \,dx $\\
        $\frac{x}{4} - (\frac{1}{4})\int {\frac{1+cos(4x)}{2}}$\\
        $\frac{x}{8} - (\frac{1}{8})\frac{sin(4x)}{4}$\\\\
        Alternatively...\\
        \item $\int sin^2(x)cos^2(x) \,dx$\\
        $\int {(sinxcosx)^2} \,dx$\\
        $\frac{1}{4}\int {sin^2(2x)} \,dx$\\
        $\frac{1}{4}\int \frac{1-cos(4x)}{2} \,dx$\\
        $\frac{x}{8} - (\frac{1}{8})\frac{sin(4x)}{4}$\\\\

        \item $\int e^{3y}\sqrt{1-e^{2y}} \,dy$\\
        $\int (e^{2y}\sqrt{1-e^{2y}})e^{y} \,dy$  Let: $u=e^{y}$\\
        $\int (u^2\sqrt{1-u^2}) \,dy$ Let: $u=sin{\theta}; \theta=sin^{-1}(y)$\\
        $\int {sin^2\theta}{cos^2\theta} \,d\theta$\\
        $\frac{\theta}{8} - \frac{sin(4\theta)}{32} + C$\\
        $\frac{sin^{-1}(y)}{8} - \frac{2sin(2\theta)cos(2\theta)}{32} + C$\\
        $\frac{sin^{-1}(y)}{8} - \frac{(2(sin\theta)(cos\theta))(cos^2\theta-sin^2\theta)}{16} + C$\\
        $\frac{sin^{-1}(y)}{8} - (\frac{1}{8})(e^{y}\sqrt{1-e^{2y}}(1-e^{2y}-e^{2y})) + C$\\
        
        \item $\int {sin(3x)cos(5x)} \,dx$\\
        Let: $u=sin(3x);du=3cos(3x)dx;dv=cos(5x)dx;v=\frac{sin(5x)}{5}$\\
        $\frac{1}{5}(sin(3x)cos(5x)) - \frac{3}{5}\int {sin(5x)cos(3x)} \,dx$\\
        Let: $a=cos(3x);da=-3sin(3x)dx;db=sin(5x)dx;b=-\frac{cos(5x)}{5}$\\
        $\frac{1}{5}(sin(3x)cos(5x)) - \frac{3}{5} [-\frac{1}{5}cos(3x)cos(5x)- \frac{3}{5}\int {sin(3x)cos(5x)}]$\\
        $\frac{34}{5}\int {sin(3x)cos(5x)} \,dx = \frac{1}{5}(sin(3x)cos(5x)) + \frac{3}{25}cos(3x)cos(5x)$\\
        $\int {sin(3x)cos(5x)} \,dx = [\frac{1}{5}(sin(3x)cos(5x)) + \frac{3}{25}cos(3x)cos(5x)]\frac{5}{34} +C$\\

        Alternatively... Remember that:\\
        sin(A+B)=sinAcosB+cosAsinB\\
        sin(A-B)=sinAcosB-cosAsinB\\
        sin(A+B)sin(A-B)=2sinAcosB\\
        In the example: A=3x; B=5x\\
        Easier to solve with it:\\
        $\frac{1}{2}\int sin(8x)-sin(2x) \,dx$\\
        $\frac{-cos(8x)}{16}+\frac{cos(2x)}{4}$\\
    \end{enumerate}
\end{enumerate}
\end{document}