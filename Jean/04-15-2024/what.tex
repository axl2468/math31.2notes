\documentclass{article}
\usepackage{amsmath}
\usepackage{amsfonts}
\usepackage{amssymb}
\usepackage{multicol}
\usepackage{graphicx}

\graphicspath{ {./images/} }

\setlength{\parindent}{0pt}

\begin{document}

\section*{04/01/2024}
\subsection*{PHYS31.01: Analytical Physics I}
        \begin{enumerate}
            \item \textbf{Examples:}
            \begin{enumerate}
                \item \textbf{Example 2:}
                Imagine a uniform ladder of weight 200[N] lying on a frictionless wall and a rough floor of certain $\mu_s$. The distance of the bottom of the ladder to the wall is 3[m], while the distance of the top of the ladder to the floor is 4[m].
                \begin{enumerate}
                    \item What is the $\mu_s$ of the floor so that the ladder is at static equilibrium? 
                    \textit{Conditions: }
                    \begin{align*}
                        \sum_i\vec{F}_i&=0 \\
                        \sum_i\vec{\tau}_i&=0 \\
                    \end{align*}
                        \textit{Along vertical: }
                    \begin{align*}
                        N_f-W&=0 \\
                        N_f&=200[N] \\
                        \\
                        \sum_i\vec{\tau}_i&=0 \\
                        \vec{\tau}_{\omega}+\vec{\tau}_{N}+\vec{\tau}_{fs}&=0 \\
                        |\vec{\tau}_{fs}|&=Lf_ssin\phi=(Lsin\phi)f_s \\
                        &=(4[m])f_s \\
                        |\vec{\tau}_{W}|&=(1.5[m])W \\
                        |\vec{\tau}_{N_f}|&=(3[m])N_f \\
                        0&=-(4[m])f_s-(1.5[m])W+(3[m])N_f \\
                        f_s&=\frac{3N_f-1.5W}{4} \\
                        f_s&=\frac{3W(1-1/2)}{4}=75[N] \\
                        \\
                        f_s&=\mu_sN \\
                        75[N]&=\mu_s(200[N]) \\
                        \mu_s&=3/8 \\
                        \\
                    \end{align*} 
                    \item What is the force exerted by the wall on the ladder?
                    \begin{align*}
                        N_w-f_s&=0 \\
                        N_w&=75[N] \\
                        \\
                    \end{align*}
                \end{enumerate}
                \item \textbf{Example 3:}
                A wheel that has a mass $m$ and radius $R$ is trying to climb a ledge at point $O$ of height $h$ with a force $|\vec{F}|$ applied on top of the wheel. 
                \begin{enumerate}
                    \item What should be the minimum $|\vec{F}|$ so that the wheel can make its landing on the ledge?
                    \\
                    \textit{We apply: $\sum_i\vec{\tau}_i=0$ at pt. 0} \\
                    \textit{We try to get d, where d is the distance textit of the wheel rotation axis about point O from point O. Thus, by trigonometry:} \\
                    \begin{align*}
                        d&=aa \\
                        \\
                        \vec{\tau}_mg+\vec{F}&=0 \\
                        mg\sqrt{2Rh-h^2}-F(2R-h)&=0 \\
                        \\
                    \end{align*}
                    \textit{It is minimum force to climb because N is already assumed to be 0. Thus...} \\
                    \begin{align*}
                        F_{min}&=mg\frac{\sqrt{2Rh-h^2}}{2R-h} \\
                        \\
                    \end{align*} 
                    \item What is the force exerted by the ledge on the wheel at point of contact $O$.
                    \begin{align*}
                        |\vec{F}_r|&=\sqrt{F_{r_ix}^2+F_{r_iy}^2} \\
                        \\
                        V:&\vec{F}_{r_iy}-mg=0 \\
                        \Rightarrow\vec{F}_{r_iy}&=mg \\
                        H:&-\vec{F}_{r_ix}+F_{min}=0 \\
                        \Rightarrow\vec{F}_{r_ix}&=mg\frac{\sqrt{2Rh-h^2}}{2R-h} \\
                        |\vec{F}_r|&=mg\sqrt{1+\frac{2Rh-h^2}{(2R-h)^2}} \\
                        \\
                    \end{align*} 
                \end{enumerate}
            \end{enumerate}
            \item \textbf{Young's Modulus}
            -Young's modulus Y where:
            \begin{align*}
                Y&=\frac{stress}{strain} \\
                stress&=\frac{\vec{F}_{\perp}}{A}&(\textit{SI unit is $[N/m^2]$ or $Pa$}) \\
                strain&=\frac{\Delta l}{l_0} \\
                &=\frac{l_f-l_0}{l_0} \\
            \end{align*}
        \end{enumerate}
\end{document}