\documentclass{article}
\usepackage{amsmath}
\usepackage{amsfonts}
\usepackage{amssymb}
\usepackage{multicol}
\usepackage{graphicx}
%\usepackage{hyperref}%

\graphicspath{ {./images/} }

\setlength{\parindent}{0pt}

\begin{document}
\section*{04/29/2024}
\subsection*{PHYS31.01: Analytical Physics 1}
\begin{enumerate}
    \item \textbf{Equation of Motion of Linear Oscillation} \\
    \begin{align*}
        a+\omega_f^2x&=constant \\
        a&:\textit{ acceleration} \\
        \omega_f^2&:\textit{ proportionality constant} \\
        x&:\textit{ displacement of object about equilibrium point} \\
    \end{align*}
    \item \textbf{Some Formulas} \\
    \begin{align*}
        \textit{Linear Oscillation} \\
        a=-\omega_f^2x \\
        v_{max}&=\omega_fA \\
        \\
        \textit{Rotational Oscillation} \\
        \alpha&=-\omega_f^2\theta \\
        \omega_{max}&=\omega_f^2A \\
    \end{align*}
    \item \textbf{Equation of Motion of Linear Oscillation} \\
    \textbf{Summary of Oscillation Parameters:} \\
    Amplitude: maximum angular displacement \\
    Period: time to complete one cycle \\
    Phase: indicates lead or lag of oscillation \\
    INC \\
    \item \textbf{Plots} \\
    Just some calculus of trig functions lol \\
    \item \textbf{Phase Constant} \\
    When phase constant is  \textit{zero}, the position is \textit{max} and velocity is \textit{zero}. \\
    When phase constant is  \textit{negative}, the position \\
    When phase constant is  \textit{positive}, the position \\
    INC \\
    \item \textbf{Problems Involving SHM} \\
    \textit{\textbf{Linear Oscillation (Horizontal)}} \\
    \\
    Recall: Hooke's Law \\
    $F_{spring}=-kx$ \\
    How?? \\
    \begin{align*} 
        ma&=-kx \\
        a&=-\frac{k}{m}x \\
        &Recall: a=-\omega_f^2x\\
        \omega_f^2&=\frac{k}{m} \\
        \omega_f&=\sqrt{\frac{k}{m}} \\
        \\
        f&=\frac{\omega_f}{2\pi}=\frac{1}{2\pi}\sqrt{k/m} \\
        Period(T_{per})&=2\pi\sqrt{m/k} \\
        \\
        \textit{Linear Osci} \\
        a+\frac{k}{m}x&=0 \\
        \\
        \textit{Linear Osci} \\
        a+\frac{k}{m}x&=0 \\
        \\
        \textit{Linear Osci} \\
        a+\frac{k}{m}x&=0 \\
    \end{align*}
    \textit{\textbf{Linear Oscillation (Vertical)}} \\
    \begin{align*}
        a+\frac{k}{m}y&=-g \\
        -ky-mg&=ma \\
        \\
        \omega_f&=\sqrt{k/m} \\ 
    \end{align*}
    \textit{\textbf{Rotational Oscillation (Simple Pendulum)}} \\
    The gravitational force does torque on a simple pendulum \\
    \\
    Equation of Motion: \\
    \begin{align*}
        -mgLsin\theta&=I\alpha=mL^2\alpha \\
        -gsin\theta&=L\alpha \\
        sin\theta\approx\alpha &\textit{(for small oscillations)} \\
        \alpha+\frac{g}{L}\theta&=0 \\
    \end{align*}
    Solution: \\
    \begin{align*} 
        \theta(t)&=\theta_0cos(\omega_ft+\delta) \\
    \end{align*}
    Angular Frequency: \\
    \begin{align*} 
        \omega_f&=\sqrt{\frac{g}{L}} \\
    \end{align*}
    \textit{\textbf{Rotational Oscillation (Physical Pendulum)}} \\
    The gravitational force does torque on a physical pendulum \\
    \\
    Equation of Motion: \\
    \begin{align*}
        ?? \\
    \end{align*}
    Solution: \\
    \begin{align*} 
        \theta(t)&=\theta_0cos(\omega_ft+\delta) \\
    \end{align*}
    Angular Frequency: \\
    \begin{align*} 
        \omega_f&=\sqrt{\frac{MgL}{I}} \\
    \end{align*}
    \\
    \\
\end{enumerate}
\subsection*{Math 10: Mathematics in the Modern World}
\begin{enumerate}
    \item \textbf{Centrality} \\
    - a concept in graph theory used to measure the \textit{importance} of certain vertices in a graph. It can be described in a variety of ways. \\
    - assigns a numerical value to a vertex that helps you compare vertices in your netwrok in terms of importance and criticality. \\
    \begin{enumerate}
        \item \textbf{Degree Centrality} \\
        - the higher the degree, the more important a node is (larger immediate connections). \\
        Ex: (Social media connections) \\
        \item \textbf{Closeness Centrality} \\
        - the \textit{geodesic distance} between two vertices $u$ and $v$ is the number of edges, if any, on a shortest path between $u$ and $v$. If the vertices are not connected by a path, their geodesic distance is $\inf$. \\
        - the \textit{closeness centrality} of a vertex is the sum of the geodesic distance between that vertex and all the other vertices in the network, i.e. \\
        \begin{align*} 
            C(u)&=\sum_{i=1}^n d(u,v_i) \\
        \end{align*}
        - vertices with high closeness centrality are those that can access the rest of the vertices with \underline{less work.}
        \item \textbf{Betweenness Centrality} \\
        - measures the importance of a vertex in a network based upon how many times it occurs in the shortest path between all pairs of vertices in a graph. \\
        - It is given as follows:
        \begin{align*} 
            B(u)&=\sum_{a,b\in V(G)} \frac{\textit{no. of shortest paths between $a$ and $b$, passing through $u$}}{\textit{no. of shortest paths between $a$ and $b$}} \\
        \end{align*}
        - Vertices with high betweenness centrality are important because their absence or removal in the network would \textit{disrupt} the paths between many pairs of vertices in the network. \\
    \end{enumerate}
    \item 
\end{enumerate}
\end{document}