\documentclass{article}
\usepackage{amsmath}
\usepackage{amsfonts}
\usepackage{amssymb}
\usepackage{multicol}
\usepackage{graphicx}

\graphicspath{ {./images/} }

\setlength{\parindent}{0pt}

\begin{document}

\section{Integration by Parts}

\begin{equation*}
    \int u \, dv = uv - \int v \, du
\end{equation*}

\section{Integration of Trigonometric Functions}

\subsection*{Moar antiderivatives}
\begin{align*}
    &&\int \sec(x)\,dx = \ln|\sec(x)+\tan(x)| + C \\
    &&\int \csc(x)\,dx = \ln|\csc(x)-\tan(x)| + C
\end{align*}

\subsection*{Trigonometric Substitution}

\begin{align*}
   &\int f(\sqrt{1-x^2}) \,dx && x = \sin\theta \text{ or } \cos\theta \\
   &\int f(\sqrt{a^2+x^2}) \,dx && x = a\tan\theta \\
   &\int f(\sqrt{a^2-x^2}) \,dx && x = a\sin\theta \text{ or } a\cos\theta \\
   &\int f(x\sqrt{x^2-a^2}) \,dx && x = a\sec\theta \\
\end{align*}

\subsection*{Useful properties/identities}
If you see $\int \sin(A)\cos(B) \, dx$, remember that
\begin{align*}
    \sin(A+B) &= \sin A\cos B + \sin A\cos B \\
    \sin(A-B) &= \sin A\cos B - \sin A\cos B \\
\end{align*}
That means that
\begin{align*}
    \sin(A+B) + sin(A-B) &= 2\sin A\cos B\\
\end{align*}

If you see $\int \sin^a(A)\cos^b(A) \, dx$, remember that
\begin{align*}
    \sin^2(A) + \cos^2(A) &= 1\\
\end{align*}
and
\begin{align*}
    \cos^2(2A) &= 2\cos^2(A) -1
\end{align*}
so
\begin{align*}
    \cos^2(A) &= \frac{1+\cos(2A)}{2} \\
    \sin^2(A) &= \frac{1-\cos(2A)}{2}
\end{align*}

If you see any integral with $\tan^a(A)$ and $\sec^b(A)$, remember that
\begin{align*}
    \tan^2(A) + 1 &= \sec^2(A)\\
\end{align*}

\begin{align*}
    &\int f(\sin{x}, \cos{x}) dx \\
    &=\int f(\sin{x})\cos{x} dx = \int f(u) du, u = \sin{x} \\\\
    &\int f(\sec{x}, \tan{x}) dx \\
    &= \int f(\cos{x})\sin{x} dx = -\int f(u) du, u = \cos{x} \\\\ 
    &\int f(\csc{x}, \cot{x}) dx \\
    &= \int \cos^5{x}\sin^3{x}dx = \int \cos^5{x}\sin^2{x}\sin{x}dx \\
    &= \int \cos^5{x}(1-\cos^2{x})\sin{x} dx \\
    &= -\int u^5(1-u^2) du, u = \cos{x} \\
    &= -\frac{u^6}{6} + \frac{u^8}{8} + c \\\\
    &\int f(\tan{x})\sec^2{x} dx = \int f(u) du, u = \tan{x} \\\\
    &\int f(\sec{x})\sec{x}\tan{x} dx = \int f(u)du, u = \sec{x} \\\\
    &\int \sec^4{x}dx = \int \sec^2{x}\sec^2{x} dx \\
    &= \int (1+\tan^2{x}) \sec^2{x} dx \\
    &= \int (1+u^2)du = u+ \frac{u^3}{3} + c, u = \tan{x} \\\\
    &\int \sec{x} dx = \ln{\sec{x}+\tan{x}} + c \\\\
    I &= \int \sec^3{x} dx = uv - \int v du \\
    &\text{Through IBP: } u=\sec{x}, du=\sec{x}\tan{x}dx, v = \tan{x}, dv = \sec^2{x}dx \\
    &= \int \sec{x}\tan{x} - \int\sec{x}\tan^2{x} dx \\
    &= \int \sec{x}\tan{x} - \int\sec{x}(\sec^2{x}-1) dx \\
    &= \int \sec{x}\tan{x} - \int\sec^3{x} dx - \int \sec{x} dx \\
    &= \int \sec{x}\tan{x} + \int\sec{x}dx - I \\
    I &= \frac{\sec{x}\tan{x}+|\sec{x}+\tan{x}|}{2} + C\\\\
    &\int\tan^3{x} dx \\
    &= \int\tan{x}\tan^2{x} dx \\
    &= \int\tan{x}(\sec^2{x}-1) dx \\
    &= \int\tan{x}\sec^2{x} dx - \int\tan{x} dx \\
    &= \int u du - ln|\sec{x}| \\
    &= \frac{\tan^2{x}}{2} - ln|\sec{x}| + C \\\\
    &\int \frac{x^2}{\sqrt{1-x^2}} dx \\
    %&=\int \frac{\sin^2{\theta}}{\cos{\theta}}\cos{\theta} d\theta, x= \sin{\theta}, dx = \cost{\theta} d\theta
\end{align*}

\end{document}

